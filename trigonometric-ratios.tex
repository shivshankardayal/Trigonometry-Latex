\chapter{Trigonometric Ratios}
From Geometry, we know that an acute angle is an angle whose measure is between $0^\circ$ and $90^\circ$. Consider the following
figure:
\begin{center}
  \begin{tikzpicture}
    \draw (0, 0) -- (4, 0);
    \draw (0, 0) -- (4, 4);
    \draw (2, 2) -- (2, 0);
    \draw (3, 3) -- (3, 0);
    \draw [blue] (0,0) node[anchor=north east] {$O$};
    \draw [blue] (2,0) node[anchor=north] {$M$};
    \draw [blue] (3,0) node[anchor=north] {$M'$};
    \draw [blue] (2,2) node[anchor=south east] {$P$};
    \draw [blue] (3,3) node[anchor=south east] {$P'$};
  \end{tikzpicture}
\end{center}

This picture contains two similar triangles $\triangle OMP$ and $\triangle OM'P'$. We are interested in $\angle MOP$ or $\angle
M'OP'0$. In the $\triangle MOP$ and $\triangle M'OP', OP, OP'$ are called the hypotenuses i.e. sides opposite to the right angle,
$PM, P'M'$ are called perpendiculars i.e. sides opposite to the angle of interest and $OM, OM'$ are called bases i.e. the third
angle.

Hypotenuses are usually denoted by $h$, perpendiculars by $p$ and bases by $b$. Let $OM = b, OM' = b', PM=p, P'M' = p', OP = h, OP'
= h'$. Since the two triangles are similar $\therefore \frac{p}{p'} = \frac{b}{b'} = \frac{h}{h'}$. Thus rhe ratio of any two sides
is dependent purely on $\angle O$ or $\angle MOP$ or $\angle M'OP'$.

Since there are three sides, we can choose $2$ in ${}^3C_2$ i.e. $3$ ways and for each combination there will be two permutations
where a side can be in either numerator or denominator. From this we can conclude that there will be six ratios(these are called
trigonometric ratios), These six trigonometric ratios or functions are given below:

$\dfrac{MP}{OP}$ or $\dfrac{p}{h}$ is called the {\bf Sine} of the $\angle MOP$.

$\dfrac{OM}{OP}$ or $\dfrac{b}{h}$ is called the {\bf Cosine} of the $\angle MOP$.

$\dfrac{MP}{OM}$ or $\dfrac{p}{b}$ is called the {\bf Tangent} of the $\angle MOP$.

$\dfrac{OP}{MP}$ or $\dfrac{h}{p}$ is called the {\bf Cosecant} of the $\angle MOP$.

$\dfrac{OP}{OM}$ or $\dfrac{h}{b}$ is called the {\bf Secant} of the $\angle MOP$.

$\dfrac{OM}{MP}$ or $\dfrac{b}{p}$ is called the {\bf Cotangent} of the $\angle MOP$.

$1 - \cos MOP$ is called the {\bf Versed Sine} of $\angle MOP$ and $1 - \sin MOP$ is called the {\bf Coversed Sine} of $\angle
MOP$. These two are rarely used in trigonometry. It should be noted that the trigonometric ratios are all numbers. The name of the
trigonometric ratios are written for brevity $\sin MOP, \cos MOP, \tan MOP, \cot MOP, \sec MOP, {\rm cosec}~MOP, {\rm vers}~MP,
{\rm coverse}~MOP$.

\section{Relationship betweeen Trigonometric Functions or Ratios}
Let us represent the $\angle MOP$ with $\theta$, we observe from previous section that
$$\sin\theta = \dfrac{1}{\csc\theta}, \cos\theta = \frac{1}{\sec\theta}, \tan\theta =
\frac{1}{\cos\theta}, {\rm cosec}\theta = \frac{1}{\sin\theta}, \sec\theta = \frac{1}{\cos\theta}, \cot\theta =
\frac{1}{\tan\theta}$$

We also observe that $\tan\theta = \dfrac{\sin\theta}{\cos\theta}$ and $\cot\theta = \dfrac{\cos\theta}{\sin\theta}$

From Pythagora theorem in geometry, we know that ${\rm hypotenuse}^2 = {\rm perpendicular}^2 + {\rm base}^2$ or $h^2 = p^2 + b^2$

\begin{enumerate}
\item Dividing both side by $h^2$, we get

$$ \dfrac{p^2}{h^2} + \dfrac{b^2}{h^2} = 1$$
$$ \sin^2\theta + \cos^2\theta = 1$$

We can rewrite this as $\sin^2\theta = 1 - \cos^2\theta, \cos^2\theta = 1 - \sin^2\theta, \sin\theta = \sqrt{1 - \cos^2\theta},
\cos\theta = \sqrt{1 - \sin^2\theta}$.

\item If we divide both sides by $b^2$, then we get

$$ \frac{h^2}{b^2} = \frac{p^2}{b^2} + 1 $$
$$ \sec^2\theta = \tan^2\theta + 1$$

We can rewrite this as $\sec^2\theta - \tan^2\theta = 1, \tan^\theta = \sec^2\theta - 1, \sec\theta = \sqrt{1 + \tan^2\theta},
\tan\theta = \sqrt{sec^2\theta - 1}$

\item Similalry, if we divide by $p^2$, then we get

$$ \frac{h^2}{p^2} = 1 + \frac{b^2}{p^2}$$
$$ {\rm cosec}^2\theta = 1 + \cos^2\theta$$

We can rewrite this as ${\rm cosec}^2\theta - \cot^2\theta = 1, \cot^2\theta = {\rm cosec}^2\theta - 1, {\rm cosec}\theta = \sqrt{1 +
\cot^2\theta}, \cot\theta = \sqrt{{\rm cosec}^2\theta - 1}$
\end{enumerate}

\section{Problems}
Prove the following:

\begin{enumerate}
\item $\sqrt{\frac{1 - \cos A}{1 + \cos A}} = {\rm cosec} A - \cot A$.
\item $\sqrt{\sec^2A + {\rm cosec}^2A} = \tan A + \cot A$.
\item $({\rm cosec} A - \sin A)(\sec A - \cos A)(\tan A + \cot A) = 1$.
\item $\cos^4 A - \sin^4 A + 1 = 2\cos^2 A$.
\item $(\sin A + \cos A)(1 - \sin A\cos A) = \sin^3A + \cos^3A$.
\item $\frac{\sin A}{1 + \cos A}+\frac{1 + \cos A}{\sin A} = 2{\rm cosec} A$.
\item $\sin^6A - cos^6A = 1 - 3\cos^2A\sin^2A$.
\item $\sqrt{\frac{1 - \sin A}{1 + \sin A}} = \sec A - \tan A$.
\item $\frac{{\rm cosec} A}{{\rm cosec} A - 1} + \frac{{\rm cosec} A}{{\rm cosec} A + 1} = 2\sec^2 A$.
\item $\frac{{\rm cosec} A}{\tan A + \cot A} = \cos A$.
\item $(\sec A + \cos A)(\sec A - \cos A) = \tan^2 A + \sin^2A$.
\item $\frac{1}{\tan A + \cot A} = \sin A\cos A$.
\item $\frac{1 - \tan A}{1 + \tan A} = \frac{\cot A - 1}{\cot A + 1}$.
\item $\frac{1 + \tan^2A}{1 + \cot^2A} = \frac{\sin^2A}{\cos^2A}$.
\item $\frac{\sec A - \tan A}{\sec A + \tan A} = 1 - 2\sec A\tan A + 2\tan^2 A$.
\item $\frac{1}{\sec A - \tan A} = \sec A + \tan A$.
\item $\frac{\tan A}{1 - \cot A} + \frac{\cot A}{1 - \tan A} = \sec A{\rm cosec} A+ 1$.
\item $\frac{\cos A}{1 - \tan A} + \frac{\sin A}{1 - \cot A} = \sin A + \cos A$.
\item $(\sin A + \cos A)(\tan A + \cot A) = \sec A + {\rm cosec} A$.
\item $\sec^4A - \sec^2A = \tan^4A + \tan^2A$.
\item $\cot^4A + \cot^2A = {\rm cosec}^4A - {\rm cosec}^2A$.
\item $\sqrt{{\rm cosec}^2A - 1} = \cos A{\rm cosec} A$.
\item $\sec^2A{\rm cosec}^2A = \tan^2A + \cot^2A + 2$.
\item $\tan^2A - \sin^2A = \sin^4A \sec^2A$.
\item $(1 + \cot A - {\rm cosec} A)(1 + \tan A + \sec A) = 2$.
\item $\frac{\cot A\cos A}{\cot A + \cos A} = \frac{\cot A - \cos A}{\cot A \cos A}$.
\item $\frac{\cot A + \tan B}{\cot B + \tan A} = \cot A \tan B$.
\item $\left(\frac{1}{\sec^2 A - \cos^2A} + \frac{1}{{\rm cosec}^2A - \sin^2A}\right)\cos^2A\sin^2A = \frac{1 - \cos^2A\sin^2A}{2 +
    \cos^2A\sin^2A}$.
\item $\sin^8A - \cos^8A = (\sin^2A - \cos^2A)(1 - 2\sin^2A\cos^2A)$.
\item $\frac{\cos A{\rm cosec} A - \sin A\sec A}{\cos A + \sin A} = {\rm cosec} A - \sec A$.
\item $\frac{1}{{\rm cosec} A - \cot A} - \frac{1}{\sin A} = \frac{1}{\sin A} - \frac{1}{{\rm cosec} A + \cot A}$.
\item $\frac{\tan A + \sec A - 1}{\tan A - \sec A + 1} = \frac{1 + \sin A}{\cos A}$.
\item $(\tan A + {\rm cosec} B)^2 - (\cot B - \sec A)^2 = 2\tan A\cot B({\rm cosec} A + \sec B)$.
\item $2\sec^2 A - \sec^4A - 2{\rm cosec}^2A + {\rm cosec}^4A = \cot^4A - \tan^4A$.
\item $(\sin A + {\rm cosec} A)^2 + (\cos A + \sec A)^2 = \tan^2A + \cot^2A + 7$.
\item $({\rm cosec} A + \cot A)(1 - \sin A) - (\sec A + \tan A)(1 - \cos A) = ({\rm cosec} A - \sec A)[2 - (1 - \cos A)(1 - \sin A)]$.
\item $(1 + \cot A + \tan A)(\sin A - \cos A) = \frac{\sec A}{{\rm cosec}^2A} - \frac{{\rm cosec} A}{\sec^2A}$.
\item $\frac{1}{\sec A - \tan A} - \frac{1}{\cos A} = \frac{1}{\cos A} - \frac{1}{\sec A + \tan A}$.
\item $3(\sin A - \cos A)^4 + 4(\sin^6 A + \cos^6 A) + 6(\sin A + \cos A)^2 = 13$.
\item $\sqrt{\frac{1 + \cos A}{1 - \cos A}} = {\rm cosec} A + \cot A$.
\item $\frac{\cos A}{1 + \sin A} + \frac{\cos A}{1 - \sin A} = 2\sec A$.
\item $\frac{\tan A}{\sec A - 1} + \frac{\tan A}{\sec A + 1} = 2{\rm cosec} A$.
\item $\frac{1}{1 - \sin A} - \frac{1}{1 + \sin A} = 2\sec A\tan A$.
\item $\frac{1 + \tan^2 A}{1 + \cot^2 A} = \left(\frac{1 - \tan A}{1 - \cot A}\right)^2$.
\item $1 + \frac{2\tan^2 A}{\cos^2 A} = \tan^4 A + sec^4 A$.
\item $(1 - \sin A - \cos A)^2 = 2(1 - \sin A)(1 - \cos A)$.
\item $\frac{\cot A + {\rm cosec} A - 1}{\cot A - {\rm cosec} A + 1} = \frac{1 + \cos A}{\sin A}$.
\item $(\sin A + \sec A)^2 + (\cos A + {\rm cosec} A)^2 = (1 + \sec A{\rm cosec} A)^2$.
\item $\frac{2\sin A\tan A(1 - \tan A) + 2\sin A\sec^2A}{(1 + \tan A)^2} = \frac{2\sin A}{1 + \tan A}$.
\item If $2\sin A = 2 - \cos A,$ find $\sin A$.
\item If $8\sin A = 4 + \cos A,$ find $\sin A$.
\item If $\tan A + \sec A = 1.5,$ find $\sin A$.
\item If $\cot A + {\rm cosec} A = 5,$ find $\cos A$.
\item If $3\sec^4 A + 8 = 10\sec^2A,$ find the value of $\tan A$.
\item If $\tan^2A + \sec A = 5,$ find $\cos A$.
\item If $\tan A + \cot A = 2,$ find $\sin A$.
\item If $\sec^2A = 2 + 2\tan A,$ find $\tan A$.
\item If $\tan A = \frac{2x(x + 1)}{2x + 1},$ find $\sin A$ and $\cos A$.
\item If $3\sin A + 5\cos A = 5,$ show that $5\sin A - 3\cos A = \pm 3$.
\item If $\sec A + \tan A = \sec A - \tan A$ prove that each side is $\pm 1$.
\item If $\frac{\cos^4 A}{\cos^2 B} + \frac{\sin^4 A}{\sin^2 B} = 1,$ prove that
  \begin{enumerate}
    \item $\sin^4A + \sin^4B = 2\sin^2A \sin^2B$,
    \item $\frac{\cos^4 B}{\cos^2 A} + \frac{\sin^4 B}{\sin^2 A} = 1$.
  \end{enumerate}
\item If $\cos A + \sin A = \sqrt{2}\cos A,$ prove that $\cos A - \sin A = \pm \sqrt{2}\sin A$.
\item If $a\cos A - b\sin A = c,$ prove that $a\sin A + b\cos A = \sqrt{a^2 + b ^2 - c^2}$.
\item If $1 - \sin A = 1 + \sin A,$ then prove that value of each side is $\pm \cos A$.
\item If $\sin^4 A + \sin^2 A = 1,$ prove that
  \begin{enumerate}
    \item $\frac{1}{\tan^4 A} + \frac{1}{\tan^2A} = 1$,
    \item $\tan^4A - \tan^2 = 1$.
  \end{enumerate}
\item If $\cos^2A - \sin^2 A = \tan^2 B,$ prove that $2\cos^2B - 1 = \cos^2B - \sin^2B = \tan^2A$.
\item If $\sin A + {\rm cosec} A = 2,$ then prove that $\sin^nA + {\rm cosec}^nA = 2$.
\item If $\tan^2A = 1 - e^2$, prove that $\sec A + \tan^3A {\rm cosec}A = (2 - e^2)^{\frac{3}{2}}$.
\item Eliminate $A$ between the equations $a\sec A + b\tan A + c = 0$ and $p\sec A + q\tan A + r = 0$.
\item If ${\rm cosec} A - \sin A = m$ and $\sec A - \cos A = n,$ elimiate $A$.
\item Is the equation $\sec^2 A = \frac{4xy}{(x + y)^2}$ possible for real values of $x$ and $y$?.
\item Show that the equation $\sin A = x + \frac{1}{x}$ is imossible for real values of $x$.
\item If $\sec A - \tan A = p, p\neq 0,$ find $\tan A, \sec A$ and $\sin A$.
\item If $\sec A = p + \frac{1}{4p},$ show that $\sec A + \tan A = 2p$ or $\frac{1}{2p}$.
\item If $\frac{\sin A}{\sin B} = p, \frac{\cos A}{\cos B} = q,$ find $\tan A$ and $\tan B$.
\item If $\frac{\sin A}{\sin B} = \sqrt{2}, \frac{\tan A}{\tan B}= \sqrt{3},$ find $A$ and $B$.
\item If $\tan A + \cot A = 2,$ find $\sin A$.
\item If $m = \tan A + \sin A$ and $n = \tan A - \sin A,$ prove that $m^2 - n^2 = 4\sqrt{mn}$.
\item If $\sin A + \cos A = m$ and $\sec A + {\rm cosec} A = n,$ prove that $n(m^2 - 1) = 2m$.
\item If $x\sin^3 A + y\cos^3 A = \sin A\cos A$ and $x\sin A - y\cos A = 0,$ prove that $x^2 + y^2 = 1$.
\item Prove that $\sin^2A = \frac{(x + y)^2}{4xy}$ is possible for real values of $x$ and $y$ only when $x =
    y$ and $x,y \neq 0$.
\end{enumerate}

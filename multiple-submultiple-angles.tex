\chapter{Multiple and Submultiple Angles}
\section{Multiple Angles}
An angle of the form $nA,$ where $n$ is an integer is called a \textit{multiple angle}. For example, $2A, 3A, 4A,
\ldots$ are multiple angles of $A.$

\subsection{Trigonometrical Ratios of $2A$}
From previous chapter we know that $\sin(A + B) = \sin A\cos B + \cos A\sin B$

\noindent Substituting $B = A$, we get $\sin 2A = 2\sin A\cos A$

\noindent Similarly, $\cos 2A = \cos^2A - \sin^2A = 2\cos^2A -  1 = 1 - 2\sin^2A$ (recall formula from previous chapter and
substitute $B = A$ $\cos^2A = 1 -\sin^2A$ and $\sin^2A = 1 - \cos^2a$)

\noindent Also, $\tan 2A = \frac{2\tan A}{1 - \tan^2A}$ (recall formula from previous chapter and put $B = A$)

\subsection{$\sin 2A$ and $\cos 2A$ in terms of $\tan A$}
$\sin 2A = \frac{2\sin A\cos A}{\sin^2A + \cos^2A}[\because \sin^2A + \cos^2A = 1]$

\noindent Dividing both numerator and denominator by $\cos^2A,$ we get

\noindent $\sin 2A = \frac{2\tan A}{1 + \tan^2A}$

\noindent $\cos A = \cos^2A - \sin^2A = \frac{\cos^2A - \sin^2A}{\cos^2A + \sin^2A}[\because \sin^2A + \cos^2A = 1]$

\noindent Dividing both numerator and denominator by $\cos^2A,$ we get

\noindent $\cos 2A = \frac{1 - \tan^2A}{1 + \tan^2A} = \frac{\cot^2A - 1}{\cot^2A + 1}$

\subsection{Trigonometrical Ratios of $3A$}
$\sin 3A = \sin2A\cos A + \cos 2A\sin A = 2\sin A\cos^2 A + \cos^2A\sin A - \sin^3A$

\noindent $= 2\sin A(1 - \sin^2A) + (1 - 2\sin^2A)\sin A - \sin^3A = 3\sin A - 4\sin^3A$

\noindent $\cos 3A = \cos2A\cos A - \sin 2A\sin A = (2\cos^2A - 1)\cos A - 2\sin^2 A\cos A$

\noindent $= 2\cos^3A - \cos A - 2(1 - \cos^2A)\cos A = 4\cos^3 A - 3\cos A$

\noindent We know that $\tan(A + B + C) = \frac{\tan A + \tan B + \tan C - \tan A\tan B\tan C}{1 - \tan A\tan B - \tan B\tan C -
  \tan C\tan A}$

\noindent Putting $B = A$ and $C = A$, we get $\tan 3A = \frac{3\tan A - \tan^3A}{1 - 3\tan^2A}$

\noindent Similarly, $\cot 3A = \frac{\cot^3 A - 3\cot A}{3\cot^2A - 1}$

\section{Some Important Formulae}
\begin{enumerate}
  \item $\cos2A = 1 - 2\sin^2A \Rightarrow \sin^2A = \frac{1}{2}(1 - \cos2A)$
  \item $\cos2A = 2\cos^2 - 1 \Rightarrow \cos^2A = \frac{1}{2}(1 + \cos2A)$
  \item $\sin 3A = 3\sin A - 4\sin^3A \Rightarrow \sin^3A = \frac{1}{2}(3\sin A - \sin3A)$
  \item $\cos 3A = 4\cos^3A - 3\cos A \Rightarrow \cos^3A = \frac{1}{4}(\cos3A + 3\cos A)$
\end{enumerate}

\section{Submultiple Angles}
An angle of the form $\frac{A}{n},$ where $n$ is an integer is called a \textit{submultiple angle}. For exmaple,
$\frac{A}{2}, \frac{A}{3}, \frac{A}{4}, \ldots$ are submultiple angles of $A.$

\subsection{Trigonometrical Ratios of $A/2$}
We know that, $\sin 2A = 2\sin A\cos A.$ Putting $A=A/2,$ we get $\sin A = 2\sin A/2\cos A/2$

\noindent $\cos 2A = \cos^2A - \sin^A.$ Putting $A = A/2,$ we get $\cos A - \cos^2\frac{A}{2} - \sin^2\frac{A}{2}$

\noindent $\cos 2A = 2\cos^2A - 1.$ Putting $A = A/2,$ we get $\cos A = 2\cos^2\frac{A}{2} - 1$

\noindent $\cos 2A = 1 - 2\sin^2A.$ Putting $A = A/2,$ we get $\cos A = 1 - 2\sin^2\frac{A}{2}$

\noindent $\tan 2A = \frac{2\tan A}{1 - \tan^2A}.$ Putting $A = A/2,$ we get $\tan A = \frac{2\tan \frac{A}{2}}{1 - \tan^2\frac{A}{2}}$

\noindent $\sin 2A = \frac{2\tan A}{1 + \tan^2A} \therefore \sin A = \frac{2\tan \frac{A}{2}}{1 + \tan^2\frac{A}{2}}$, $\cos 2A = \frac{1 - \tan^2A}{1 + \tan^2A} \therefore \cos A = \frac{1 - \tan^2\frac{A}{2}}{1 + \tan^2\frac{A}{2}}$

\noindent $\cot 2A = \frac{\cot^2A - 1}{2\cot A} \therefore \cot A = \frac{\cot^2\frac{A}{2} - 1}{2\cot \frac{A}{2}}$

\subsection{Trigonometrical Ratios of $A/3$}
$\sin 3A = 3\sin A - 4\sin^3A.$ Putting $A = \frac{A}{3},$ we get $\sin A = 3\sin \frac{A}{3} - 4\sin^3\frac{A}{3}$

\noindent $\cos 3A = 4\cos^3A - 3\cos A$. Putting $A = \frac{A}{3},$ we get $\cos A = 4\cos^3\frac{A}{3} - 3\cos \frac{A}{3}$

$\tan 3A = \frac{3\tan A - \tan^3A}{1 - 3\tan^2A}$ $\Rightarrow \tan A = \frac{3\tan\frac{A}{3} - \tan^3\frac{A}{3}}{1 - 3\tan^2\frac{A}{3}}$

\subsection{Values of $cos A/2, \sin A/2$ and $\tan A/2$ in terms of $\cos A$}
$\cos^2\frac{A}{2} = \frac{1 + \cos A}{2} \therefore \cos \frac{A}{2} = \sqrt{\frac{1 + \cos A}{2}}$

\noindent $\sin^2\frac{A}{2} = \frac{1 - \cos A}{}2 \therefore \sin \frac{A}{2} = \sqrt{\frac{1 - \cos A}{2}}$

\noindent $\tan^2\frac{A}{2} = \frac{1 - \cos A}{1 + \cos A} \therefore \tan\frac{A}{2} = \sqrt{\frac{1 - \cos A}{1 + \cos A}}$

\subsection{Values of $\sin A/2$ and $\cos A/2$ in terms of $\sin A$}
$\left(\cos \frac{A}{2} + \sin\frac{A}{2}\right)^2 = \cos^2\frac{A}{2} + \sin^2\frac{A}{2} + 2\cos\frac{A}{2}\sin\frac{A}{2}$

\noindent $= 1 + \sin A \Rightarrow \cos \frac{A}{2} + \sin \frac{A}{2} = \sqrt{1 + \sin a}$

\noindent Similarly, $\cos \frac{A}{2} - \sin \frac{A}{2} = \sqrt{1 - \sin a}$

\noindent Adding, we get $\cos \frac{A}{2} = \pm\frac{1}{2}\sqrt{1 + \sin A} \pm\frac{1}{2}\sqrt{1 - \sin a}$

\noindent Subtracting, we get $\cos \frac{A}{2} = \pm\frac{1}{2}\sqrt{1 + \sin A} \mp\frac{1}{2}\sqrt{1 - \sin A}$

\subsection{Value of $\sin 18^\circ$ and $\cos 72^\circ$}
Let $A = 18^\circ,$ then $\sin 5A = 90^\circ \therefore 2A + 3A = 90^\circ$ $\Rightarrow \sin2A = \sin(90^\circ - \sin 3A) \therefore 2\sin A\cos A = 4\cos^3A - 3\cos A$

\noindent Dividing both sides by $\cos A,$ we get $2\sin A = 4\cos^2A - 3 = 4(1 - \sin^2A) - 3$ $\Rightarrow 4\sin^2A + 2\sin A - 1 = 0$ $\Rightarrow \sin A = \frac{-1\pm\sqrt{5}}{4}$

\noindent However, since $A= 18^\circ\therefore \sin A > 0$ $\therefore \sin18^\circ = \frac{-1 + \sqrt{5}}{4}$ $\therefore \sin(90^\circ - 18^\circ) = \cos72^\circ = \frac{\sqrt{5} - 1}{4}$

\subsection{Value of $\cos 18^\circ$ and $\sin 72^\circ$}
$\cos^218^\circ = 1 - \sin^218^\circ = 1 - \left(\frac{\sqrt{5} - 1}{4}\right)^2 = \frac{10 + 2\sqrt{5}}{16}\therefore \cos18^\circ = \frac{1}{4}\sqrt{10 + 2\sqrt{5}}[\because \cos18^\circ > 0]$

\noindent $\cos(90^\circ - 18^\circ) = \sin72^\circ = \frac{1}{4}\sqrt{10 + 2\sqrt{5}}$

\subsection{Value of $\tan 18^\circ$ and $\tan 72^\circ$}
$\tan 18^\circ = \frac{\sin18^\circ}{\cos18^\circ} = \frac{\sqrt{5} - 1}{\sqrt{10 + 2\sqrt{5}}}$

\noindent $\tan18^\circ\cot18^\circ = 1\Rightarrow \tan72^\circ = \frac{1}{\tan18^\circ} = \frac{\sqrt{10 + 2\sqrt{5}}}{\sqrt{5} - 1}$

\subsection{Value of $\cos 36^\circ$ and $\sin 54^\circ$}
$\cos 36^\circ = 1 - 2\sin^218^\circ = 1 - 2\left(\frac{\sqrt{5} - 1}{4}\right)^2= \frac{\sqrt{5} + 1}{4}$

\noindent $\sin 54^\circ = \sin(90^\circ - 36^\circ) = \cos36^\circ = \frac{\sqrt{5} + 1}{4}$

\subsection{Value of $\sin 36^\circ$ and $\cos 54^\circ$}
$\sin36^\circ = 1 - \cos^236^\circ = 1 - \left(\frac{\sqrt{5} + 1}{4}\right)^2 = \frac{1}{4}\sqrt{10 - 2\sqrt{5}}$

\noindent$\cos54^\circ = \cos(90^\circ - 36^\circ) = \sin36^\circ = \frac{1}{4}\sqrt{10 - 2\sqrt{5}}$

\noindent Several other angles like, $9^\circ, 15^\circ, 22\frac{1}{2}^\circ, 7\frac{1}{2}^\circ$ etc can be found similarrly.

\section{Problems}
\begin{enumerate}
\item Find the value of $\sin 2A$, when
  \begin{enumerate}
  \item $\cos A = \frac{3}{5}$.
  \item $\sin A = \frac{12}{13}$.
  \item $\tan A = \frac{16}{63}$.
  \end{enumerate}
\item Find the value of $\cos 2A$, when
  \begin{enumerate}
  \item $\cos A = \frac{15}{17}$.
  \item $\sin A = \frac{4}{5}$.
  \item $\tan A = \frac{5}{12}$.
  \end{enumerate}
\item If $\tan A = \frac{b}{a}$, find the value of $a\cos 2A+ b\sin 2A$.
\end{enumerate}

Prove that

\begin{enumerate}[resume]
\item $\frac{\sin 2A}{1 + \cos 2A} = \tan A$.
\item $\frac{\sin 2A}{1 - \cos 2A} = \cot A$.
\item $\frac{1 - \cos 2A}{1 + \cos 2A} = \tan^2A$.
\item $\tan A + \cot A = 2{\rm cosec} 2A$.
\item $\tan A - \cot A = -2\cot2A$.
\item ${\rm cosec} 2A + \cot 2A = \cot A$.
\item $\frac{1 - \cos A + \cos B - \cos(A + B)}{1 + \cos A - \cos B - \cos(A + B)} = \tan\frac{A}{2}\cot\frac{B}{2}$.
\item $\frac{\cos A}{1 \mp \sin A} = \tan\left(45^\circ \pm \frac{A}{2}\right)$.
\item $\frac{\sec 8A - 1}{\sec 4A - 1} = \frac{\tan 8A}{\tan 2A}$.
\item $\frac{1 + \tan^2(45^\circ - A)}{1 - \tan^2(45^\circ - A)} = {\rm cosec} 2A$.
\item $\frac{\sin A + \sin B}{\sin A - \sin B} = \frac{\tan \frac{A + B}{2}}{\tan \frac{A - B}{2}}$.
\item $\frac{\sin^2A - \sin^2B}{\sin A\cos A - \sin B\cos B} = \tan(A + B)$.
\item $\tan\left(\frac{\pi}{4} + A\right) - \tan\left(\frac{\pi}{4} - A\right) = 2\tan 2A$.
\item $\frac{\cos A + \sin A}{\cos A - \sin A} - \frac{\cos A - \sin A}{\cos A + \sin A} = 2\tan 2A$.
\item $\cot (A + 15^\circ) - \tan(A - 15^\circ) = \frac{4\cos 2A}{1 + 2\sin 2A}$.
\item $\frac{\sin A + \sin2A}{1 + \cos A + \cos 2A} = \tan A$.
\item $\frac{1 + \sin A - \cos A }{1 + \sin A + cos A} = \tan \frac{A}{2}$.
\item $\frac{\sin(n + 1)A - \sin(n - 1)A}{\cos(n + 1)A + 2\cos nA + \cos(n - 1)A} = \tan \frac{A}{2}$.
\item $\frac{\sin(n + 1)A + 2\sin nA + \sin(n - 1)A}{\cos(n - 1) - \cos(n + 1)A} = \cot \frac{A}{2}$.
\item $\sin(2n + 1)A\sin A = \sin^2(n + 1)A - \sin^2nA$.
\item $\frac{\sin(A + 3B) + \sin(3A + B)}{\sin 2A + \sin 2B} = 2\cos(A + B)$.
\item $\sin 3A + \sin 2A - \sin A = 4\sin A\cos \frac{A}{2}\cos \frac{3A}{2}$.
\item $\tan 2A = (\sec 2A + 1)\sqrt{\sec^2A - 1}$.
\item $\cos^32A + 3\cos 2A = 4(\cos^6A - \sin^6A)$.
\item $1 + \cos^22A = 2(\cos^4A + \sin^4A)$.
\item $\sec^2A(1 + \sec2A) = 2\sec2A$.
\item ${\rm cosec} A - 2\cot 2A\cos A = 2\sin A$.
\item $\cot A = \frac{1}{2}\left(\cot\frac{A}{2} - \tan\frac{A}{2}\right)$.
\item $\sin A\sin(60^\circ - A)\sin(60^\circ + A) = \frac{1}{4}\sin 3A$.
\item $\cos A\cos(60^\circ - A)\cos(60^\circ + A) = \frac{1}{4}\cos 3A$.
\item $\cot A + \cot(60^\circ + A) - \cot(60^\circ - A) = 3\cot 3A$.
\item $\cos 4A = 1 - 8\cos^2A + 8\cos^4A$.
\item $\sin 4A = 4\sin A\cos^3A - 4\cos A\sin^3A$.
\item $\cos 6A = 32\cos^6A - 48\cos^4A + 18\cos^2A - 1$.
\item $\tan 3A\tan 2A\tan A = \tan 3A - \tan 2A - \tan A$.
\item $\frac{2\cos2^nA + 1}{2\cos A + 1} = (2\cos A - 1)(2\cos 2A - 1)(2\cos2^2A - 1)\ldots(2\cos2^{n - 1} - 1)$.
\item If $\tan A= \frac{1}{7}, \sin B = \frac{1}{\sqrt{10}},$ prove that $A + 2B = \frac{\pi}{4},$ where $0 < A <
    \frac{\pi}{4}$ and  $0 < B < \frac{\pi}{4}$.
\end{enumerate}

Prove that

\begin{enumerate}[resume]
\item $\tan\left(\frac{\pi}{4} + A\right) + \tan\left(\frac{\pi}{4} - A\right) = 2\sec2A$.
\item $\sqrt{3}{\rm cosec} 20^\circ - \sec 20^\circ = 4$.
\item $\tan A + 2\tan 2A + 4\tan 4A + 8\cot 8A = \cot A$.
\item $\cos^2A + \cos^2\left(\frac{2\pi}{3} - A\right) + \cos^2\left(\frac{2\pi}{3} + A\right) = \frac{3}{2}$.
\item $2\sin^2A + 4\cos (A + B)\sin A\sin B + \cos2(A + B)$ is idnependent of $A$.
\item If $\cos A = \frac{1}{2}\left(a + \frac{1}{a}\right),$ show that $\cos 2A = \frac{1}{2}\left(a^2 +
  \frac{1}{a^2}\right)$.
\end{enumerate}

Prove that

\begin{enumerate}[resume]
\item $\cos^2A + \sin^2A\cos 2B = \cos^2B + \sin^2B\cos 2A$.
\item $1 + \tan A\tan 2A = \sec 2A$.
\item $\frac{1 + \sin 2A}{1 - \sin 2A} = \left(\frac{1 + \tan A}{1 - \tan A}\right)^2$.
\item $\frac{1}{\sin 10^\circ} - \frac{\sqrt{3}}{\cos 10^\circ} = 4$.
\item $\cot^2A - \tan^2A = 4\cot2A{\rm cosec} 2A$.
\item $\frac{1 +\sin 2A}{\cos2A} = \frac{\cos A + \sin A}{\cos A - \sin A} = \tan\left(\frac{\pi}{4} + A\right)$.
\item $\cos^6A - \sin^6A = \cos2A\left(1 - \frac{1}{4}\sin^22A\right)$.
\item $\cos^2A + \cos^2\left(\frac{\pi}{3} + A\right) + \cos^2\left(\frac{\pi}{3} - A\right)= \frac{3}{2}$.
\item $(1 + \sec2A)(1 + \sec2^2A)(1 + \sec2^3A) \ldots (1 + \sec2^nA) = \frac{\tan2^nA}{\tan A}$.
\item $\frac{\sin2^nA}{\sin A} = 2^n\cos A\cos 2A\cos 2^2A\ldots\cos2^{n - 1}A$.
\item $3(\sin A - \cos A)^4 + 6(\sin A + \cos A)^2 + 4(\sin^6A + \cos^6A) = 13$.
\item $2(\sin^6A + \cos^6A) - 3(\sin^4A + \cos^4A) + 1 = 0$.
\item $\cos^2A + \cos^2(A + B) -2\cos A\cos B\cos(A + B)$ if independent of $A$.
\item $\cos^3A\cos 3A + \sin^3A\sin 3A = \cos^32A$.
\item $\tan A\tan(60^\circ - A)\tan(60^\circ + A) = \tan 3A$.
\item $\sin^2A + \sin^3\left(\frac{2\pi}{3} + A\right) + \sin^3\left(\frac{4\pi}{3} + A\right) = -\frac{3}{4}\sin 3A$.
\item $4(\cos^310^\circ + \sin^320^\circ) = 3(\cos 10\circ + \sin 20^\circ)$.
\item $\sin A\cos^3A - \cos A\sin^3A = \frac{1}{4}\sin 4A$.
\item $\cos^3A\sin3A + \sin^3A\cos 3A = \frac{3}{4}\sin 4A$.
\item $\sin A\sin(60^\circ + A)\sin(A + 120^\circ) = \sin 3A$.
\item $\cot A + \cot(60^\circ + A) + \cot(120^\circ + A) = 3\cot 3A$.
\item $\cos 5A = 16\cos^5A - 20\cos^3A + 5\cos A$.
\item $\sin 5A = 5\sin A - 20\sin^3A + 16\sin^5A$.
\item $\cos 4A - \cos 4B = 8(\cos A - \cos B)(\cos A + \cos B)(\cos A - \sin B)(\cos A + \sin B)$.
\item $\tan 4A = \frac{4\tan A - 4\tan^3A}{1 - 6\tan^2A + \tan^4A}$.
\item If $2\tan A = 3\tan B,$ prove that $\tan (A- B) = \frac{\sin 2B}{5 - \cos 2B}$.
\item If $\sin A + \sin B = x$ and $\cos A + \cos B = y,$ show that $\sin(A + B) = \frac{2xy}{x^2 + y^2}$.
\item If $A= \frac{\pi}{2^n + 1},$ prove that $\cos A.\cos 2A. \cos2^2A.\ldots.\cos2^{n - 1}A = \frac{1}{2^n}$.
\item If $\tan A = \frac{y}{x},$ prove that $x\cos 2A + y\sin 2A = x$.
\item If $\tan^2A = 1 + 2\tan^2B,$ prove that $\cos 2B = 1 + 2\cos 2A$.
\item If $A$ and $B$ lie between $0$ and $\frac{\pi}{2}$ and $\cos 2A = \frac{3\cos 2B - 1}{3 - \cos
  2B},$ prove that $\tan A = \sqrt{2}\tan B$.
\item If $\tan B = 3\tan A,$ prove that $\tan(A + B) = \frac{2\sin 2B}{1 + \cos 2B}$.
\item If $x\sin A = y\cos A,$ prove that $\frac{x}{\sec 2A} + \frac{y}{{\rm cosec} 2A} = x$.
\item If $\tan A = \sec 2B,$ prove that $\sin 2A = \frac{1 - \tan^4B}{1 + \tan^4B}$.
\item If $A = \frac{\pi}{3},$ prove that $\cos A.\cos 2A. \cos 3A.\cos 4A.\cos 5A.\cos 6A = -\frac{1}{16}$.
\item If $A = \frac{\pi}{15},$ prove that $\cos2A.\cos4A.\cos8A.\cos14A = \frac{1}{16}$.
\item If $\tan A\tan B = \sqrt{\frac{a - b}{a + b}},$ prove that $(a - b\cos2A)(a - b\cos2B) = a^2 - b^2$.
\item If $\sin A = \frac{1}{2}$ and $\sin B = \frac{1}{3},$ find the value of $\sin(A + B)$ and $\sin(2A +
  2B)$.
\item If $\cos A = \frac{11}{61}$ and $\sin B = \frac{4}{5},$ find the value of $\sin^2 \frac{A - B}{2}$ and
  $cos^2\frac{A + B}{2},$ the angle of $A$ and $B$ being positive acute angles.
\item Given $\sec A = \frac{5}{4},$ find $\tan\frac{A}{2}$ and $\tan A$.
\item If $\cos A = .3,$ find the value of $\tan \frac{A}{2},$ and explain the resulting ambiguity.
\item If $\sin A + \sin B = x$ and $\cos A + \cos B = y,$ find the value of $\tan \frac{A - B}{2}$.
\end{enumerate}

Prove that

\begin{enumerate}[resume]
\item $(\cos A + \cos B)^2 + (\sin A - \sin B)^2 = 4\cos^2 \frac{A + B}{2}$.
\item $(\cos A + \cos B)^2 + (\sin A + \sin B)^2 = 4\cos^2 \frac{A - B}{2}$.
\item $(\cos A - \cos B)^2 + (\sin A - \sin B)^2 = 4\sin^2 \frac{A - B}{2}$.
\item $\sin^2\left(\frac{\pi}{8} + \frac{A}{2}\right) - \sin^2\left(\frac{\pi}{8} -\frac{A}{2}\right) = \frac{1}{\sqrt{2}}\sin
  A$.
\item $(\tan 4A + \tan 2A)(1 - \tan^23A\tan^2A) = 2\tan 3A\sec^2A$.
\item $\left(1 + \tan \frac{A}{2} - \sec\frac{A}{2}\right)\left(1 + \tan \frac{A}{2} + \sec\frac{A}{2}\right) = \sin
  A\sec^2\frac{A}{2}$.
\item $\frac{1 + \sin A - \cos A}{1 + \sin A + \cos A} = \tan \frac{A}{2}$.
\item $\frac{1 - \tan \frac{A}{2}}{1 + \tan \frac{A}{2}} = \frac{1 + \sin A}{\cos A} = \tan \left(\frac{\pi}{4} +
  \frac{A}{2}\right)$.
\item $\cos^4\frac{\pi}{8} + \cos^4 \frac{3\pi}{8} + \cos^4\frac{5\pi}{8} + \cos^4\frac{7\pi}{8}= \frac{3}{2}$.
\item $\frac{2\sin A - \sin2A}{2\sin A + \sin 2A} = \tan^2\frac{A}{2}$.
\item $\cot \frac{A}{2} - \tan \frac{A}{2} = 2\cot A$.
\item $\frac{1 + \sin A}{1 - \sin A} = \tan^2\left(\frac{\pi}{4} + \frac{A}{2}\right)$.
\item $\sec A + \tan A = \tan\left(\frac{\pi}{4} + \frac{A}{2}\right)$.
\item $\frac{\sin A + \sin B - \sin(A + B)}{\sin A + \sin B + \sin(A + B)} = \tan \frac{A}{2}\tan \frac{B}{2}$.
\item $\tan \left(\frac{\pi}{4} - \frac{A}{2}\right) = \sec A - \tan A = \sqrt{\frac{1 - \sin A}{1 + \sin A}}$.
\item ${\rm cosec}\left(\frac{\pi}{4} + \frac{A}{2}\right){\rm cosec} \left(\frac{\pi}{4} - \frac{A}{2}\right) = 2\sec A$.
\item $\cos^2\frac{\pi}{8} + \cos^2\frac{3\pi}{8} + \cos^2\frac{5\pi}{8} + \cos^2\frac{7\pi}{8} = 2$.
\item $\sin^4\frac{\pi}{8} + \sin^4 \frac{3\pi}{8} + \sin^4\frac{5\pi}{8} + \sin^4\frac{7\pi}{8} = \frac{3}{2}$.
\item $\left(1 + \cos \frac{\pi}{8}\right)\left(1 + \cos\frac{3\pi}{8}\right)\left(1 + \cos\frac{5\pi}{8}\right)\left(1 + \cos
  \frac{7\pi}{8}\right) = \frac{1}{8}$.
\item Find the value of $\sin \frac{23\pi}{24}$.
\item If $A = 112^\circ30',$ find the value of $\sin A$ and $\cos A$.
\end{enumerate}

Prove that

\begin{enumerate}[resume]
\item $\sin^224^\circ - \sin^26^\circ = \frac{1}{8}(\sqrt{5} - 1)$.
\item $\tan6^\circ.\tan42^\circ.\tan66^\circ.\tan78^\circ = 1$.
\item $\sin47^\circ + \sin61^\circ - \sin 11^\circ - \sin25^\circ = \cos 7^\circ$.
\item $\sin 12^\circ\sin48^\circ\sin54^\circ = \frac{1}{8}$.
\item $\cot 142\frac{1}{2}^\circ = \sqrt{2} + \sqrt{3} - 2 - \sqrt{6}$.
\item $\sin^248^\circ - \cos^212^\circ = -\frac{\sqrt{5} + 1}{8}$.
\item $4(\sin 24^\circ + \cos6^\circ) = \sqrt{3} + \sqrt{15}$.
\item $\cot6^\circ\cot42^\circ\cot66^\circ\cot78^\circ = 1$.
\item $\tan12^\circ\tan24^\circ\tan48^\circ\tan84^\circ = 1$.
\item $\sin6^\circ\sin42^\circ\sin66^\circ\sin78^\circ = \frac{1}{16}$.
\item $\sin\frac{\pi}{5}\sin\frac{2\pi}{5}\sin\frac{3\pi}{5}\sin\frac{4\pi}{5} = \frac{5}{16}$.
\item $\cos36^\circ\cos72^\circ\cos108^\circ\cos144^\circ = \frac{1}{16}$.
\item $\cos\frac{\pi}{15}\cos\frac{2\pi}{15}\cos\frac{3\pi}{15}\cos\frac{4\pi}{15}\cos\frac{5\pi}{15}\cos\frac{6\pi}{15}\cos\frac{7\pi}{15}
  = \frac{1}{2^7}$.
\item $\cos\frac{\pi}{65}\cos\frac{2\pi}{65}\cos\frac{4\pi}{65}\cos\frac{8\pi}{65}\cos\frac{16\pi}{65}\cos\frac{32\pi}{65} =
  \frac{1}{64}$.
\item If $\tan \frac{A}{2} = \sqrt{\frac{a - b}{a + b}}\tan \frac{B}{2},$ prove that, $\cos A = \frac{a\cos B + b}{a +
  b\cos B}$.
\item If $\tan \frac{A}{2} \ = \sqrt{\frac{1 - e}{1 + e}}\tan\frac{B}{2},$ prove that $\cos B = \frac{\cos A - e}{1 -
  e\cos A}$.
\item If $\sin A + \sin B = a$ and $\cos A + \cos B = b,$ prove that $\sin(A + B) = \frac{2ab}{a^2 + b^2}$.
\item If $\sin A + \sin B = a$ and $\cos A + \cos B = b,$ prove that $\cos(A - B) = \frac{1}{2}(a^2 + b^2 - 2)$.
\item If $A$ and $B$ be two different roots of equation $a\cos\theta + b\sin\theta = c,$ prove that
  \begin{enumerate}
  \item $\tan(A + B) = \frac{2ab}{a^2 - b^2}$.
  \item $\cos(A + B) = \frac{a^2 - b^2}{a^2 + b^2}$.
  \end{enumerate}
\item If $\cos A + \cos B = \frac{1}{3}$ and $\sin A + \sin B = \frac{1}{4},$ prove that $\cos \frac{A - B}{2} =
  \pm\frac{5}{24}$.
\item If $2\tan \frac{A}{2} = \tan \frac{B}{2},$ prove that $\cos A = \frac{3 + 5\cos B}{5 + 3\cos B}$.
\item If $\sin A = \frac{4}{5}$ and $\cos B = \frac{5}{13},$ prove that one value of $\cos \frac{A - B}{2} =
  \frac{8}{\sqrt{65}}$.
\item If $\sec(A + B) + \sec(A - B) = 2\sec A,$ prove that $\cos B = \pm \sqrt{2}\cos \frac{B}{2}$.
\item If $\cos \theta = \frac{\cos\alpha\cos\beta}{1 - \sin\alpha\sin\beta},$ prove that one of the values of $\tan
     \frac{\theta}{2}$ is $\frac{\tan \frac{\alpha}{2} - \tan\frac{\beta}{2}}{1 - \tan\frac{\alpha}{2}\tan\frac{\beta}{2}}$.
\item If $\tan\alpha = \frac{\sin\theta\sin\phi}{\cos\theta + \cos\phi},$ prove that one of the values of
     $\tan\frac{\alpha}{2}$ is $\tan\frac{\theta}{2}\tan\frac{\phi}{2}$.
\item If $\cos\theta = \frac{\cos\alpha + \cos\beta}{1 + \cos\alpha\cos\beta},$ prove that one of the values of
     $\tan\frac{\theta}{2}$ is $\tan\frac{\alpha}{2}\tan\frac{\beta}{2}$.
\end{enumerate}
